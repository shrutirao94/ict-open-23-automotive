%%%% Small single column format, used for CIE, CSUR, DTRAP, JACM, JDIQ, JEA, JERIC, JETC, PACMCGIT, TAAS, TACCESS, TACO, TALG, TALLIP (formerly TALIP), TCPS, TDSCI, TEAC, TECS, THRI, TIIS, TIOT, TISSEC, TIST, TKDD, TMIS, TOCE, TOCHI, TOCL, TOCS, TOCT, TODAES, TODS, TOIS, TOIT, TOMACS, TOMM (formerly TOMCCAP), TOMPECS, TOMS, TOPC, TOPLAS, TOPS, TOS, TOSEM, TOSN, TRETS, TSAS, TSC, TSLP, TWEB.
% \documentclass[format=acmsmall, review=false, screen=true]{acmart}

%%%% Large single column format, used for IMWUT, JOCCH, PACMPL, POMACS, TAP, PACMHCI
\documentclass[acmlarge]{acmart}
%%%% Large double column format, used for TOG
% \documentclass[acmtog, authorversion]{acmart}

%%%% Generic manuscript mode
%\documentclass[manuscript, review, screen]{acmart}
%\setcitestyle{super,sort&compress}
%\citestyle{acmnumeric}
\usepackage{booktabs} % For formal tables
%\usepackage{hyperref}

%\usepackage[ruled]{algorithm2e} % For algorithms
%\usepackage{listings}
%\usepackage{glossaries} % For abbreviations
%\usepackage[acronym,nonumberlist]{glossaries-extra}

%\loadglsentries{acronym_list}


% Load basic packages
%\usepackage{balance}       % to better equalize the last page
%\usepackage{graphics}      % for EPS, load graphicx instead
%\usepackage[T1]{fontenc}   % for umlauts and other diaeresis
%\usepackage{txfonts}
%\usepackage{mathptmx}
%\usepackage[pdflang={en-US},pdftex]{hyperref}
%%\usepackage{color}
%%\usepackage{booktabs}
%%\usepackage{textcomp}
%%\usepackage{gensymb}



% Package for algorithmic code
\usepackage{algorithm}
\usepackage[noend]{algpseudocode}
\usepackage{subcaption}
\makeatletter
\def\BState{\State\hskip-\ALG@thistlm}
\makeatother


% Metadata Information
%\acmJournal{IMWUT}
%\acmVolume{0}
%\acmNumber{0}
%\acmArticle{0}
%\acmYear{2018}
%\acmMonth{3}

%\acmBadgeL[http://ctuning.org/ae/ppopp2016.html]{ae-logo}
%\acmBadgeR[http://ctuning.org/ae/ppopp2016.html]{ae-logo}

% Copyright
%\setcopyright{acmcopyright}
%\setcopyright{acmlicensed}
%\setcopyright{rightsretained}
%\setcopyright{usgov}
\setcopyright{usgovmixed}
%\setcopyright{cagov}
%\setcopyright{cagovmixed}

% DOI
%\acmDOI{0000001.0000001}


% Document starts
\begin{document}


% Title portion
\title{Investigating Affective Responses toward In-Video Pedestrian Crossing Actions using Camera and Physiological Sensors}

\author{Shruti Rao}
\affiliation{%
  \institution{Distributed \& Interactive Systems, Centrum Wiskunde \& Informatica}
  % \city{Science Park 123, 1098 XG, Amsterdam}
  \country{The Netherlands}}
\email{s.rao@cwi.nl}

\author{Surjya Ghosh}
\affiliation{%
  \institution{Distributed \& Interactive Systems, Centrum Wiskunde \& Informatica}
  % \city{Science Park 123, 1098 XG, Amsterdam}
  \country{The Netherlands}}
\email{surjya.ghosh@gmail.com}


\author{Gerard Pons}
\affiliation{
 \institution{Distributed \& Interactive Systems, Centrum Wiskunde \& Informatica}
 \city{Amsterdam}
 \country{The Netherlands}}
\email{gerardponsr@gmail.com}

\author{Thomas R\"{o}ggla}
\affiliation{
    \institution{Distributed \& Interactive Systems, Centrum Wiskunde \& Informatica}
    \city{Amsterdam}
    \country{The Netherlands}}
\email{t.roggla@cwi.nl}


\author{Abdallah El Ali}
\affiliation{
    \institution{Distributed \& Interactive Systems, Centrum Wiskunde \& Informatica}
    \city{Amsterdam}
    \country{The Netherlands}}
\email{aea@cwi.nl}

\author{Pablo Cesar}
\affiliation{
    \institution{Distributed \& Interactive Systems, Centrum Wiskunde \& Informatica}
    \institution{Delft University of Technology}
    \city{Amsterdam}
    \country{The Netherlands}}
\email{p.s.cesar@cwi.nl}


\renewcommand\shortauthors{Rao et al.}


\keywords{empathic car, pedestrian non-verbal behaviour, driver emotion recognition, physiological sensing, thermal sensing}

\maketitle

\section*{Abstract}
\textbf{Motivation:} 
\textit{Empathic} cars \footnote{https://www.theguardian.com/business/2018/jan/23/a-car-which-detects-emotions-how-driving-one-made-us-feel}\footnote{https://www.irishtimes.com/business/transport-and-tourism/researchers-developing-empathic-car-technology-1.3900701} are being developed to identify driver emotions during driving scenarios. This is because emotions (particularly anger or stress) that arise during driving scenarios are known to adversely impact driving behaviour \cite{2015:tf:jeon}. While environmental and situational (traffic) factors have been considered for inferring drivers' emotional states \cite{2018:frontiers:habibovic, 2019:MTI:braun, jeon2016don}, the non-verbal interaction between a driver and pedestrian(s) has received less attention. Considering that pedestrian non-verbal behaviour is often a source of negative driver emotion \cite{zepf2019towards}, automatically capturing drivers' affective responses toward pedestrian non-verbal actions can aid in designing empathic, in-vehicle interfaces towards increased road safety. To investigate the influence of pedestrian, non-verbal crossing actions, we adopted a controlled experimental approach using physiological and camera sensors. Specifically, we ask: \textit{How do people's affective responses vary in response to different non-verbal, pedestrian crossing actions shown through video stimuli?}
\\


\noindent \textbf{User Study:}  
We conducted an in-lab study where participants with driving experience (\textit{N=21}) watched $10$ short videos of driving scenarios from the \textit{Joint Attention for Autonomous Driving (JAAD)} dataset \cite{2017:IV:rasouli, Ghosh2022}. These videos show pedestrians crossing the road and performing non-verbal actions towards the driver such as hand waving, nodding etc \cite{2016:arxiv:jaad}. For each video, participants rated pedestrian actions for valence and arousal using the 9-point discrete Self-Assessment Manikin (SAM) \cite{bradley1994measuring}. Additionally, participants' facial temperatures were recorded using a FLIR Duo Pro R camera. Tparticipants' pupil diameter was captured using a Pupil Labs eye tracker, and physiological signals of heart rate and skin conductance were collected using an Empatica E4 wristband. Recruited participants \textit{(7f, 14m)} were between 22-64 years of age \textit{(M=32.4, SD=11.6)}, and comprised diverse cultural backgrounds (66\% European, 24\% Asian, and 10\% North American). $76\%$ of participants had at least three years of driving experience in Western Europe \textit{(M=9.8, SD=10.7)}, with no reported visual, auditory, or motor impairments.
\\

\noindent \textbf{Results:} 
We observed that participants reported higher valence (pleasantness) upon observing positive pedestrian crossing action videos and higher arousal (excitement) upon watching non-positive pedestrian crossing videos. Additionally, participants' physiological signals (heart rate, skin conductance and pupil diameter) were significantly influenced ($p<0.05$) by the positive and non-positive pedestrian crossing actions. Finally, participants facial temperatures also vary significantly ($p<0.05$) for different levels of participants' valence (positive versus non-positive) and arousal (high versus non-high) scores. Therefore, our work offers two key contributions - \textbf{(1)}: Validation of non-verbal, pedestrian crossing stimuli (JAAD videos) that influence participants' affective states though multi-modal physiological and camera sensors. \textbf{(2)}: Empirical findings which reveal that non-verbal, pedestrian actions influence participants' self-reported emotions (valence and arousal), physiological signals and facial temperatures. 

While our study lacks ecological validity (being an in-lab controlled setup), and does not address the robustness of detected participant signals towards just-in-time interventions, our results may be used for the development of emotion recognition models. These models can leverage affective cues (physiological, behavioural and thermal) during driver-pedestrian interactions as part of an emotion self-regulation framework for improving road safety. Our future work will involve designing a hybrid simulator setup where participants drive in a simulator but interact with real-world pedestrian crossing actions. 



\bibliographystyle{ACM-Reference-Format}
\bibliography{report}
\end{document}
